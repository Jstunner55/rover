\documentclass[fleqn,10pt]{SelfArx} % Document font size and equations flushed left
\usepackage[english]{babel} % Specify a different language here - english by default
\usepackage{lipsum} % Required to insert dummy text. To be removed otherwise



\setlength{\columnsep}{0.55cm} % Distance between the two columns of text
\setlength{\fboxrule}{0.75pt} % Width of the border around the abstract
\definecolor{color1}{RGB}{0,0,90} % Color of the article title and sections
\definecolor{color2}{RGB}{0,20,20} % Color of the boxes behind the abstract and headings
\usepackage{hyperref} % Required for hyperlinks
\hypersetup{
	hidelinks,
	colorlinks,
	breaklinks=true,
	urlcolor=color2,
	citecolor=color1,
	linkcolor=color1,
	bookmarksopen=false,
	pdftitle={Title},
	pdfauthor={Author},
}
\PaperTitle{Project Requirements - DirtNav} % Article title
\Authors{Jose Reyes\textsuperscript{1}} % Authors
\affiliation{\textsuperscript{1}\textit{Founder, ReyMex, Bismarck, ND, USA}} % Author affiliation
\affiliation{\textsuperscript{2}\textit{Electrical Staff Engineer, Kestrel Engineering Group, Bismarck, ND, USA}} % Author affiliation
\affiliation{\textsuperscript{3}\textit{B.S. Mechanical Engineering, University of Mary, Bismarck, ND, USA}} % Author affiliation
\affiliation{\textsuperscript{4}\textit{B.S. Electrical Engineering, University of Mary, Bismarck, ND, USA}} % Author affiliation
\affiliation{*\textbf{Corresponding author}: jstunner55@gmail.com} % Corresponding author



\begin{document}

\maketitle % Output the title and abstract box
\tableofcontents % Output the contents section
\thispagestyle{empty} % Removes page numbering from the first page


\section{Project Description}

\subsection{Purpose}
The purpose of this rover is to serve as a practical outdoor utility platform for residential lawncare. 
Its primary role is to collect elevation and soil moisture data across a property at a significantly lower cost than commercially available lawncare robots.

\subsection{Scope}
The scope of this document is limited to the \textit{pre-control} aspects of rover development. 
This includes the proposed mechanical design, electrical architecture, wiring and harness routing strategy, and structural layout. 
A functional prototype (“mule build”) and all rover control software are explicitly out of scope for this phase.

\subsection{System Overview}
The rover is intended to operate year-round and must therefore withstand precipitation, temperature variation, and high winds. 
It will traverse multiple outdoor surfaces—including grass, dirt, pavement, ice, and snow—and maintain traction on varying inclines. 
The design philosophy prioritizes durability, serviceability, and functional performance, while maintaining a clean and professional appearance.

\section{Specific Requirements}

\subsection{Mechanical Requirements}
\begin{itemize}
    \item Must climb inclines up to $30^\circ$.
    \item Must operate reliably in rain, snow, and high-wind conditions.
    \item Must traverse snow, ice, dirt, pavement, and grass.
    \item Must achieve a minimum straight-line speed of 13~mph.
\end{itemize}

\subsection{Electrical Requirements}
\begin{itemize}
    \item Must provide at least 60 minutes of continuous runtime per charge.
    \item Must operate on a 12~V electrical bus.
    \item Must be controllable remotely through internet-based communication protocols.
    \item Must include actuated soil-moisture sensing capability.
    \item Must include elevation and terrain-profiling sensing capability.
\end{itemize}

\subsection{Harness Routing Requirements}
\begin{itemize}
    \item No wires may be free-hanging within the chassis.
    \item No twisted, nested, or uncontrolled wire bundles are permitted.
    \item No wiring harnesses may be visible externally when the rover is fully assembled.
\end{itemize}

\subsection{Structural Requirements}
\begin{itemize}
    \item All electronic enclosures must be fully waterproof.
    \item The rover must survive a 6~ft drop onto any face without structural or functional damage.
    \item The complete assembly must fit within a 2~ft~$\times$~2~ft~$\times$~2~ft volume.
\end{itemize}

\subsection{Functionality}
The rover shall autonomously collect and store environmental data, relay information to a remote operator, and navigate outdoor terrain using onboard sensors and a remote communication link. Additional functional requirements will be developed in later phases as control algorithms and operational behaviors are defined.

\section{Non-Functional Requirements}

\subsection{Performance Requirements}
\begin{itemize}
    \item The rover shall respond to sensor inputs and operator commands within approximately 200~ms during normal operation.
    \item The rover shall maintain stable traction across grass, dirt, pavement, snow, and ice under typical outdoor conditions.
    \item The rover shall be capable of reaching a minimum straight-line speed of 13~mph on flat pavement.
    \item The rover shall continuously collect and log moisture, elevation, and general telemetry data while in motion.
\end{itemize}

\subsection{Reliability Requirements}
\begin{itemize}
    \item The rover shall operate reliably for a full 60-minute session without unexpected shutdowns or major faults.
    \item The rover shall recover from common issues—such as temporary sensor errors or communication interruptions—without requiring a full system reboot.
    \item The rover shall function in rain, light snow, and winds up to 30~mph without loss of essential capability.
    \item The rover shall tolerate minor impacts with obstacles such as rocks, branches, or edges of pavement without sustaining damage that affects operation.
\end{itemize}

\subsection{Design Constraints}
\begin{itemize}
    \item The rover shall operate on a 12~V electrical system, and all components shall remain compatible with this power architecture.
    \item All electronics shall be enclosed in waterproof housings to ensure safe operation in outdoor environments.
    \item The complete mechanical and electrical assembly shall fit within a 2~ft $\times$ 2~ft $\times$ 2~ft volume.
    \item Wireless communication shall comply with standard FCC regulations for consumer-grade devices.
    \item All wiring shall be routed internally, with no exposed or loosely suspended harnesses on the exterior of the rover.
\end{itemize}
\end{document}